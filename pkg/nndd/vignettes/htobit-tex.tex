\documentclass[nojss]{jss}
\usepackage{booktabs,dcolumn,thumbpdf,lmodern}

\author{Achim Zeileis\\Universit\"at Innsbruck}
\Plainauthor{Achim Zeileis}

\title{Heteroscedastic Tobit Regression}

\Keywords{heteroscedastic tobit, regression, \proglang{R}}
\Plainkeywords{heteroscedastic tobit, regression, R}

\Abstract{
  The \pkg{htobit} package (\url{https://R-Forge.R-project.org/projects/uibk-rprog-2017/})
  fits tobit regression models with conditional heteroscedasticy using maximum
  likelihood estimation. A brief overview of the package is provided, along
  with some illustrations and a replication of results from the \pkg{crch} package.
}

\Address{
  Achim Zeileis\\
  Department of Statistics\\
  Faculty of Economics and Statistics\\
  Universit\"at Innsbruck\\
  Universit\"atsstr.~15\\
  6020 Innsbruck, Austria\\
  E-mail: \email{Achim.Zeileis@R-project.org}\\
  URL: \url{https://statmath.wu.ac.at/~zeileis/}
}

%% Sweave/vignette information and metadata
%% need no \usepackage{Sweave}
%\VignetteIndexEntry{Heteroscedastic Tobit Regression}
%\VignetteEngine{Sweave}
%\VignetteDepends{htobit, car, crch, lmtest, memisc}
%\VignetteKeywords{heteroscedastic tobit, regression, R}
%\VignettePackage{htobit}



\begin{document}
\input{htobit-tex-concordance}

\section{Introduction}

The heteroscedastic tobit model assumes an underlying latent Gaussian variable
\[
  y_i^* \sim \mathcal{N}(\mu_i, \sigma_i^2)
\]
which is only observed if positive and zero otherwise: $y_i = \max(0, y_i^*)$.
The latent mean $\mu_i$ and scale $\sigma_i$ (latent standard deviation)
are linked to two different linear predictors
\begin{eqnarray*}
\mu_i & = & x_i^\top \beta \\
\log(\sigma_i) & = & z_i^\top \gamma
\end{eqnarray*}
where the regressor vectors $x_i$ and $z_i$ can be set up without restrictions,
i.e., they can be identical, overlapping or completely different or just including an intercept, etc.

See also \cite{crch} for a more detailed introduction to this model class as well as
a better implementation in the package \pkg{crch}. The main purpose of \pkg{htobit} is
to illustrate how to create such a package \emph{from scratch}.


\section{Implementation}

As usual in many other regression packages for \proglang{R} \citep{R}, the main model fitting function \code{htobit()}
uses a formula-based interface and returns an (\proglang{S}3) object of class \code{htobit}:
%
\begin{verbatim}
htobit(formula, data, subset, na.action,
  model = TRUE, y = TRUE, x = FALSE,
  control = htobit_control(...), ...)
\end{verbatim}
%
Actually, the \code{formula} can be a two-part \code{Formula} \citep{Formula}, specifying separate sets of regressors
$x_i$ and $z_i$ for the location and scale submodels, respectively.

The underlying workhorse function is \code{htobit_fit()} which has a matrix interface and returns an unclassed list.

A number of standard \proglang{S}3 methods are provided, see Table~\ref{tab:methods}.

\begin{table}[t!]
\centering
\begin{tabular}{lp{11.2cm}}
\hline
Method & Description \\ \hline
\code{print()}       & Simple printed display with coefficients \\
\code{summary()}     & Standard regression summary; returns \code{summary.htobit} object (with \code{print()} method) \\
\code{coef()}	     & Extract coefficients \\
\code{vcov()}	     & Associated covariance matrix \\
\code{predict()}     & (Different types of) predictions for new data \\
\code{fitted()}      & Fitted values for observed data \\
\code{residuals()}   & Extract (different types of) residuals \\
\code{terms()}       & Extract terms \\
\code{model.matrix()}& Extract model matrix (or matrices) \\
\code{nobs()}	     & Extract number of observations \\
\code{logLik()}      & Extract fitted log-likelihood \\
\code{bread()}       & Extract bread for \pkg{sandwich} covariance \\
\code{estfun()}      & Extract estimating functions (= gradient contributions) for \pkg{sandwich} covariances \\
\code{getSummary()}  & Extract summary statistics for \code{mtable()} \\
\code{Boot()}	     & Bootstrap regression coefficients using \pkg{car} and \pkg{boot} \\ \hline
\end{tabular}
\caption{\label{tab:methods} \proglang{S}3 methods provided in \pkg{htobit}.}
\end{table}

Due to these methods a number of useful utilities work automatically, e.g., \code{AIC()}, \code{BIC()},
\code{coeftest()} (\pkg{lmtest}), \code{lrtest()} (\pkg{lmtest}), \code{waldtest()} (\pkg{lmtest}),
\code{linearHypothesis()} (\pkg{car}), \code{mtable()} (\pkg{memisc}), etc.


\section{Illustration}

To illustrate the package's use in practice, a comparison of several homoscedastic
and heteroscedastic tobit regression models is applied to
data on budget shares of alcohol and tobacco for 2724 Belgian households
\citep[taken from][]{Verbeek:2004}. The homoscedastic model from \cite{Verbeek:2004} can
be replicated by:

\begin{Schunk}
\begin{Sinput}
R> data("AlcoholTobacco", package = "htobit")
R> library("htobit")
R> ma <- htobit(alcohol ~ (age + adults) * log(expenditure) + oldkids + youngkids,
+    data = AlcoholTobacco)
R> summary(ma)
\end{Sinput}
\begin{Soutput}
Call:
htobit(formula = alcohol ~ (age + adults) * log(expenditure) + 
    oldkids + youngkids, data = AlcoholTobacco)

Standardized residuals:
    Min      1Q  Median      3Q     Max 
-1.0698 -0.4407 -0.1364  0.3934  8.3170 

Coefficients (location model):
                          Estimate Std. Error z value Pr(>|z|)    
(Intercept)             -0.1591533  0.0437782  -3.635 0.000278 ***
age                      0.0134938  0.0108824   1.240 0.214989    
adults                   0.0291901  0.0169469   1.722 0.084989 .  
log(expenditure)         0.0126679  0.0032156   3.939 8.17e-05 ***
oldkids                 -0.0026408  0.0006049  -4.366 1.27e-05 ***
youngkids               -0.0038789  0.0023835  -1.627 0.103651    
age:log(expenditure)    -0.0008093  0.0008006  -1.011 0.312067    
adults:log(expenditure) -0.0022484  0.0012232  -1.838 0.066051 .  

Coefficients (scale model with log link):
            Estimate Std. Error z value Pr(>|z|)    
(Intercept) -3.71236    0.01533  -242.1   <2e-16 ***
---
Signif. codes:  0 '***' 0.001 '**' 0.01 '*' 0.05 '.' 0.1 ' ' 1 

Log-likelihood:  4755 on 9 Df
Number of iterations in BFGS optimization: 102 
\end{Soutput}
\end{Schunk}

This model is now modified in two directions: First, the variables influencing the location
parameter are also employed in the scale submodel. Second, because the various coefficients
for different numbers of persons in the household do not appear to be very different,
a restricted specification for this is used. Using a Wald test for testing linear hypotheses
from \pkg{car} \citep{car} yields

\begin{Schunk}
\begin{Sinput}
R> library("car")
R> linearHypothesis(ma, "oldkids = youngkids")
\end{Sinput}
\begin{Soutput}
Linear hypothesis test

Hypothesis:
oldkids - youngkids = 0

Model 1: restricted model
Model 2: alcohol ~ (age + adults) * log(expenditure) + oldkids + youngkids

  Df  Chisq Pr(>Chisq)
1                     
2  1 0.2639     0.6075
\end{Soutput}
\begin{Sinput}
R> linearHypothesis(ma, "oldkids = adults")
\end{Sinput}
\begin{Soutput}
Linear hypothesis test

Hypothesis:
- adults  + oldkids = 0

Model 1: restricted model
Model 2: alcohol ~ (age + adults) * log(expenditure) + oldkids + youngkids

  Df  Chisq Pr(>Chisq)  
1                       
2  1 3.4994    0.06139 .
---
Signif. codes:  0 '***' 0.001 '**' 0.01 '*' 0.05 '.' 0.1 ' ' 1
\end{Soutput}
\end{Schunk}

Therefore, the following models are considered:

\begin{Schunk}
\begin{Sinput}
R> AlcoholTobacco$persons <- with(AlcoholTobacco, adults + oldkids + youngkids)
R> ma2 <- htobit(alcohol ~ (age + adults) * log(expenditure) + oldkids + youngkids |
+    (age + adults) * log(expenditure) + oldkids + youngkids, data = AlcoholTobacco)
R> ma3 <- htobit(alcohol ~ age + log(expenditure) + persons | age +
+    log(expenditure) + persons, data = AlcoholTobacco)
R> BIC(ma, ma2, ma3)
\end{Sinput}
\begin{Soutput}
    df       BIC
ma   9 -9439.553
ma2 16 -9735.109
ma3  8 -9777.154
\end{Soutput}
\end{Schunk}

The BIC would choose the most parsimonious model but a likelihood ratio test would prefer
the unconstrained person coefficients:

\begin{Schunk}
\begin{Sinput}
R> library("lmtest")
R> lrtest(ma, ma2, ma3)
\end{Sinput}
\begin{Soutput}
Likelihood ratio test

Model 1: alcohol ~ (age + adults) * log(expenditure) + oldkids + youngkids
Model 2: alcohol ~ (age + adults) * log(expenditure) + oldkids + youngkids | 
    (age + adults) * log(expenditure) + oldkids + youngkids
Model 3: alcohol ~ age + log(expenditure) + persons | age + log(expenditure) + 
    persons
  #Df LogLik Df   Chisq Pr(>Chisq)    
1   9 4755.4                          
2  16 4930.8  7 350.925  < 2.2e-16 ***
3   8 4920.2 -8  21.234   0.006551 ** 
---
Signif. codes:  0 '***' 0.001 '**' 0.01 '*' 0.05 '.' 0.1 ' ' 1
\end{Soutput}
\end{Schunk}

\section{Replication}

To assess the reliability of the \code{htobit()} implementation, it is benchmarked against the \code{crch()}
function of \citep{crch}, using the same restricted model as above.

\begin{Schunk}
\begin{Sinput}
R> library("crch")
R> ca3 <- crch(alcohol ~ age + log(expenditure) + persons | age +
+    log(expenditure) + persons, data = AlcoholTobacco, left = 0)
\end{Sinput}
\end{Schunk}

Using a model table from \pkg{memisc} \citep{memisc} it can be easily seen the results can be
replicated using both packages (see Table~\ref{tab:mtable}).

\begin{Schunk}
\begin{Sinput}
R> library("memisc")
R> mtable("htobit" = ma3, "crch" = ca3)
\end{Sinput}
\end{Schunk}

\begin{table}[t!]
\centering
%%%%%%%%%%%%%%%%%%%%%%%%%%%%%%%%%%%%%%%%%%%%%%%%%%%%%%%%%%%%%%%%%%%%%%%%%%%%%%%%%%%%%%%%%%%%%%%%%%%%%%%%%%%%%%%%%%%%%%%%%%%%%%%%%%%%%%%%
%
% Calls:
% htobit:  htobit(formula = alcohol ~ age + log(expenditure) + persons | age + log(expenditure) + persons, data = AlcoholTobacco) 
% crch:  crch(formula = alcohol ~ age + log(expenditure) + persons | age + log(expenditure) + persons, data = AlcoholTobacco, left = 0) 
%
%%%%%%%%%%%%%%%%%%%%%%%%%%%%%%%%%%%%%%%%%%%%%%%%%%%%%%%%%%%%%%%%%%%%%%%%%%%%%%%%%%%%%%%%%%%%%%%%%%%%%%%%%%%%%%%%%%%%%%%%%%%%%%%%%%%%%%%%
\begin{tabular}{lD{.}{.}{3}D{.}{.}{3}cD{.}{.}{3}D{.}{.}{3}}
\toprule
&\multicolumn{2}{c}{htobit}&&\multicolumn{2}{c}{crch}\\
\cmidrule{2-3}\cmidrule{5-6}
&\multicolumn{1}{c}{location}&\multicolumn{1}{c}{scale}&&\multicolumn{1}{c}{location}&\multicolumn{1}{c}{scale}\\
\midrule
(Intercept)&-0.072^{***}&0.176&&-0.072^{***}&0.176\\
&(0.014)&(0.515)&&(0.014)&(0.515)\\
age&0.002^{***}&0.064^{***}&&0.002^{***}&0.064^{***}\\
&(0.000)&(0.013)&&(0.000)&(0.013)\\
log(expenditure)&0.006^{***}&-0.278^{***}&&0.006^{***}&-0.278^{***}\\
&(0.001)&(0.038)&&(0.001)&(0.038)\\
persons&-0.002^{***}&-0.111^{***}&&-0.002^{***}&-0.111^{***}\\
&(0.000)&(0.014)&&(0.000)&(0.014)\\
\midrule
Log-likelihood&4920.2&&&4920.2&\\
AIC&-9824.4&&&-9824.4&\\
BIC&-9777.2&&&-9777.2&\\
N&2724&&&2724&\\
\bottomrule
\end{tabular}\caption{\label{tab:mtable} Replication of \pkg{crch} results using \pkg{htobit}.}
\end{table}


\bibliography{htobit}

\end{document}
